\section{Conclusion}

In summary, this work illustrates the application of the LinUCB algorithm to the KuaiRec dataset for personalized, contextualized recommendations. The dense and feature-rich KuaiRec dataset enabled a thorough exploration of the algorithm’s behavior, including its ability to learn contextual cues, balance exploration and exploitation, and achieve low regret. Even though our project did not achieve optimal results, this is believed  to be due from the complexities within the KuaiRec data set, and the simplicity of the LinUCB algorithm as a recommendation system. Nevertheless, our research is still valuable to the overall literature as it presents a framework, in applying LinUCB to the KuaiRec data set, a dense fully observed recommendation matrix, highlighting its challenges and limitations.

Although the approach has demonstrated potential in improving sequential decision-making in recommendation scenarios, the model presents heavy biases which may limit its usefulness or practicality for real-world applications. Further research should focus on understanding theses biases in this research model, developing a better metric for more accurate recommendations, and comparing the LinUCB model to other recommendation system model in order to both be able to compare different models but also identify potential areas for improvement.