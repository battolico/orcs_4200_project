\section{Data and Experimental Setup}

In this study, we utilize the KuaiRec dataset, which offers a highly dense user-item interaction matrix coupled with rich contextual information about both users and items. As described, the “small matrix” subset within KuaiRec is $99.6\%$ dense, making it particularly suitable for reliable training and evaluation. We focus on the subset due to its completeness, ensuring minimal uncertainty caused by missing data and reducing the need for extensive imputation or filtering.

\subsection{Data Description and Preprocessing}

The KuaiRec dataset comprises interactions between users and videos, along with item-specific metadata (e.g., categories, temporal information) and user-specific features (e.g., demographics, social connections, usage patterns). Each user-video interaction record includes a watch ratio, indicating the proportion of the video consumed by the user, which serves as a proxy for user engagement or reward.

To prepare the data for our contextual bandit model:

\begin{enumerate}
    \item \textbf{Data Cleaning and Integration:} We first cleaned the dataset by addressing missing values. NaN values in numerical columns were replaced by median values to preserve as many data points as possible. Non-numeric or malformed entries were either removed or assigned a placeholder value (e.g., \(-124\) for unknown categories).

    \item \textbf{Feature Engineering:} Contextual information was leveraged to form user and item feature vectors. User features (such as demographic and behavioral traits) were merged with aggregated item-level categories. We pivoted the data to produce a user-by-category matrix that represents how each user engaged with each first-level video category, \textbf{sum of liked videos} across relevant videos. These transformations allowed us to directly incorporate contextual vectors into the LinUCB framework.

    \item \textbf{Normalization and Encoding:} Where appropriate, categorical features were encoded, and time-based features were converted to numeric values representing time since the earliest observation. This step ensured that temporal sequences could be easily interpreted by the bandit model.

    \item \textbf{Train-Test Split:} We split the data into training and testing sets. The training set was used to update the LinUCB model parameters iteratively, while the testing set served to evaluate the model’s performance out-of-sample, ensuring a fair assessment of its ability to generalize.
\end{enumerate}