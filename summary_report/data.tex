\section{Data and Experimental Setup}

% In this study, we utilize the KuaiRec dataset, which offers a highly dense user-item interaction matrix coupled with rich contextual information about both users and items. As described, the “small matrix” subset within KuaiRec is $99.6\%$ dense, making it particularly suitable for reliable training and evaluation. We focus on the subset due to its completeness, ensuring minimal uncertainty caused by missing data and reducing the need for extensive imputation or filtering.

For the project, we decided to use the KuaiRec dataset \cite{gao2022kuairec}, and specifically we are going to use the small\_matrix.csv data set. The KuaiRec data set is a "real-world dataset collected from the recommendation logs of the video-sharing mobile app Kuaishou" \cite{gao2022kuairec}. What is different about this project, which uses MABs to model a recommendation system, is that the matrix data that we are going to use contains a fully observed user-item interaction matrix. In most recommendation systems, the data used to train the system are very sparse, meaning that a user has not interacted with all the possible options, and prevents us from making accurate predictions about other types of categories that the user might like. If we are building a music recommendation system, it might not be able to tell whether a user likes classical music if they only interact with rock music until now. This is an issue especially when we are dealing with a cold start problem - when a new user enters the system and we have no historical data on their preferences \cite{nath2023_medium}. Being able to conduct research with a complete user-item interaction enables us to create a strong recommendation system.


\subsection{Data Description and Preprocessing}

The KuaiRec dataset comprises interactions between users and videos, along with item-specific metadata (e.g., categories, temporal information) and user-specific features (e.g., demographics, social connections, usage patterns). Each user-video interaction record includes a watch ratio, indicating the proportion of the video consumed by the user, which can serve as a proxy for user engagement or reward.

We first start by cleaning the data set. This involves replacing NaNs, which we decide to replace by the median value. This was done in order to preserve the data, without throwing away important information. We also looked into using a KNN imputer, but that was computationally too expensive, so we opted to use the median. Non-numeric or malformed entries were either removed or assigned a placeholder value (\(-124\) for unknown categories). For the features of each user, we used 18 one-hot encoded features provided by the dataset. After investigating the data, we removed seven one-hot encoded features as they contain NaN values. Therefore, a total of 11 features are used as the context vector. The code merges the small\_matrix with two other data sets: item\_categories, and video\_categories. Then a binary feature "like" is added, based on the watch ratio column, where a watch ratio greater than 2 indicates that the video was indeed "liked". Then the data was pivoted to produce a user-by-category matrix that represents how each user engaged with each first-level video category, \textbf{sum of liked videos} across relevant videos. These transformations allowed us to directly incorporate contextual vectors into the LinUCB framework. Lastly, in order for the model to work, temporal features had to be converted to a numerical value. This was done by converting the date to the number of seconds elapsed since the lowest date time of the feature.
