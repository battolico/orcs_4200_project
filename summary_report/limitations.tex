\section{Contributions and Limitations}

This study contributes to the literature on sequential recommendation strategies by demonstrating how a contextual bandit model, specifically LinUCB, can be applied to a dense real-world dataset like KuaiRec. By leveraging rich contextual data, the LinUCB approach:

\begin{itemize}
    \item \textbf{Enhances Adaptability:} The model dynamically updates its estimates of user preferences, continually improving the relevance of its recommendations.
    \item \textbf{Balances Exploration and Exploitation:} The upper confidence bound framework encourages the algorithm to explore new categories while leveraging existing knowledge, ultimately reducing regret.
    \item \textbf{Provides a Practical Framework:} LinUCB is computationally efficient and readily interpretable, making it suitable for large-scale deployment on digital platforms.
\end{itemize}

However, there are several limitations and opportunities for future work:

\begin{itemize}
    \item \textbf{Contextual Complexity:} Although we included user and item features, real-world scenarios may demand even more complex contextual signals (e.g., temporal trends, social network influences, and cross-platform behaviors). Incorporating such complexity might require more sophisticated models, such as neural-based contextual bandits or hybrid systems.
    
    \item \textbf{Reward Definition:} We used watch ratio as a proxy for user engagement. Future studies could explore other reward metrics, such as retention, downstream conversions, or long-term user satisfaction, to better align the model’s objectives with platform goals.

    \item \textbf{Scalability and Real-Time Constraints:} While LinUCB is relatively efficient, extremely large user bases or item catalogs could still pose computational challenges. Developing incremental or approximate update strategies would be beneficial for real-time industrial applications.
    
    \item \textbf{Robustness to Non-Stationary Environments:} User preferences may shift over time, especially in rapidly changing content landscapes. Incorporating adaptive learning mechanisms or drift-detection methods could help maintain performance over extended periods.
\end{itemize}

