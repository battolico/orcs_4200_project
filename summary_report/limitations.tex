\section{Contributions and Limitations}

This study contributes to the literature on contextual MABs for recommendation system by demonstrating how LinUCB, can be applied to a dense real-world dataset like KuaiRec. The LinUCB approach balances exploration and exploitation and is also highly interpretable. Using an upper confidence bound framework, it encourages the algorithm to explore new categories while leveraging existing knowledge. With the ultimate goal of reducing regret. Unlike deep learning frameworks, LinUCB learns by tuning explainable covariance parameters which are more suitable for large-scale deployment on digital platforms, as it makes the model explainable.

Although this study included user and item features, real-world scenarios may demand even more complex contextual signals (e.g., temporal trends, social network influences, and cross-platform behaviors). Incorporating such complexity might require more sophisticated models, such as deep learning based contextual bandits or hybrid models that are both content-based and collaborative filtering recommendation systems. Moreover, while \textit{watch ratio} and \textit{likes} were used as a proxy for user engagement, future studies could explore other reward metrics, such as retention, downstream conversions, or long-term user satisfaction, to better align the model’s objectives with platform goals. Lastly, while LinUCB is relatively efficient, extremely large user bases or item catalogs could pose challenges, if an immediate response is desired. Developing incremental or approximate update strategies would be beneficial for real-time industrial applications. 
